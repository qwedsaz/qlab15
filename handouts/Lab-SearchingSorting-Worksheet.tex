\documentclass[10pt]{exam}
%\printanswers
\usepackage{fullpage}

\setlength{\parindent}{0pt}
\setlength{\parskip}{.25cm}

\usepackage{graphicx}

\usepackage{xcolor}

\definecolor{darkred}{rgb}{0.5,0,0}
\definecolor{darkgreen}{rgb}{0,0.5,0}
\usepackage{hyperref}
\hypersetup{
  letterpaper,
  colorlinks,
  linkcolor=red,
  citecolor=darkgreen,
  menucolor=darkred,
  urlcolor=blue,
  pdfpagemode=none,
  pdftitle={Introduction To Git},
  pdfauthor={Christopher M. Bourke},
  pdfcreator={$ $Id: cv-us.tex,v 1.28 2009/01/01 00:00:00 cbourke Exp $ $},
  pdfsubject={PhD Thesis},
  pdfkeywords={}
}

\definecolor{MyDarkBlue}{rgb}{0,0.08,0.45}
\definecolor{MyDarkRed}{rgb}{0.45,0.08,0}
\definecolor{MyDarkGreen}{rgb}{0.08,0.45,0.08}

\definecolor{mintedBackground}{rgb}{0.95,0.95,0.95}
\definecolor{mintedInlineBackground}{rgb}{.90,.90,1}

%\usepackage{newfloat}
\usepackage[newfloat=true]{minted}
\setminted{mathescape,
               linenos,
               autogobble,
               frame=none,
               framesep=2mm,
               framerule=0.4pt,
               %label=foo,
               xleftmargin=2em,
               xrightmargin=0em,
               startinline=true,  %PHP only, allow it to omit the PHP Tags *** with this option, variables using dollar sign in comments are treated as latex math
               numbersep=10pt, %gap between line numbers and start of line
               style=default, %syntax highlighting style, default is "default"
               			    %gallery: http://help.farbox.com/pygments.html
			    	    %list available: pygmentize -L styles
               bgcolor=mintedBackground} %prevents breaking across pages
               
\setmintedinline{bgcolor={mintedBackground}}
\setminted[text]{bgcolor={mintedBackground},linenos=false,autogobble,xleftmargin=1em}
%\setminted[php]{bgcolor=mintedBackgroundPHP} %startinline=True}
\SetupFloatingEnvironment{listing}{name=Code Sample}
\SetupFloatingEnvironment{listing}{listname=List of Code Samples}

\begin{document}

\section*{CSCE 155 - Lab 13 - Searching \& Sorting - Worksheet}

Names: \underline{\hspace{10cm}}

\begin{questions}

\question Examine the ``search, find'' section of the main function.  
  A key has been created that will match any team from California (CA).  
  Four searches are performed.
\begin{parts}
  \part What is the result of the first search?
  \begin{solution}[2cm]
  \end{solution}

  \part What is the result of the second?
  \begin{solution}[2cm]
  \end{solution}

  \part What is the result of the third?
  \begin{solution}[2cm]
  \end{solution}

  \part What is the result of the fourth (number 1B)?
  \begin{solution}[2cm]
  \end{solution}

  \part Why did the same search produce different results for all 
  three searches even though we called \mintinline{c}{bsearch} on the same inputs? 
  \begin{solution}[2cm]
  \end{solution}

  \part Observe that the fourth search (number 1B) was performed under 
    the same ordering as the first search, so why did it produce 
    a different result?
  \begin{solution}[2cm]
  \end{solution}
  
\end{parts}

\question Demonstrate your working programs to a lab instructor.  

\end{questions}
  
Lab Instructor Signature\underline{\hspace{7.5cm}}

\end{document}
